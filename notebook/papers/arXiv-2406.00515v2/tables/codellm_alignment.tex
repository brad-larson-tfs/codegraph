\begin{table}[t] 
\centering
\caption{\done{
Five core principles serve as the key objectives for Code LLMs alignment: Green, Responsibility, Efficiency, Safety, and Trustworthiness (collectively referred to as \textbf{GREST}).
}}
\label{tab:codellm_alignment}
\scalebox{0.65}{
\begin{tabular}{ll}
\toprule
\textbf{Principles} & \textbf{Involved Concepts and Properties} \\ 
\midrule
\textbf{Green} & 
\makecell[l]{
    %  \textbf{Energy Efficiency}: Designing systems to minimize energy consumption, reducing environmental impact and lowering financial costs.\\
    %  \textbf{Sustainable Materials}: Utilizing eco-friendly and recyclable materials in hardware and infrastructure to decrease long-term expenses. \\
    %  \textbf{Carbon Footprint}: Implementing practices to reduce emissions and offset carbon output, often leading to cost savings through efficiency. \\
    %  \textbf{Resource Optimization}: Efficient use of resources to prevent waste, promoting sustainability while reducing financial expenditures. \\
    %  \textbf{Recycling and E-Waste Management}: Ensuring proper disposal and recycling of electronic waste, which can reduce costs associated with waste management. \\
    %  \textbf{Renewable Energy}: Incorporating renewable energy sources to power operations, potentially lowering energy expenses. \\
    %  \textbf{Lifecycle Assessment}: Evaluating the environmental impact and financial costs of a product from creation to disposal to ensure sustainable practices.
     \textbf{Energy Efficiency}: Minimizing computational energy use and reduce environmental impact and financial costs.\\
     \textbf{Sustainable Materials}: Leveraging eco-friendly infrastructure and servers for code generation, lowering long-term expenses. \\
     \textbf{Carbon Footprint}: Reducing emissions associated with model training and inference to enhance efficiency and save costs. \\
     \textbf{Resource Optimization}: Efficiently utilizing computational resources to minimize waste and reduce expenses in code generation. \\
     \textbf{Recycling Management}: Responsibly dispose of hardware used in model development to reduce waste management costs. \\
     \textbf{Renewable Energy}: Utilizing renewable energy sources for powering training and inference processes to decrease energy costs.  \\
     \textbf{Lifecycle Assessment}: Evaluating the environmental and financial impacts of models from creation to deployment and disposal.
} \\
\midrule
\textbf{Responsibility} & 
\makecell[l]{
    %  \textbf{Privacy}: Protecting user data is a key responsibility. \\
    %  \textbf{Copyright Issue}: Involves responsibly handling intellectual property and others' work. \\
    %  \textbf{Explainability}: Enables users to understand the decision-making process, reflecting responsibility. \\
    %  \textbf{Usability}: Involves the system's user-friendliness as part of responsible design.
     \textbf{Ethical Considerations}: Adhering to ethical guidelines to ensure responsible use and deployment of generated code. \\
     \textbf{Accountability}: Establishing clear lines of responsibility for code generation outcomes and potential impacts. \\
     \textbf{User Education}: Providing resources and guidance to help users understand and responsibly use generated code. \\
     \textbf{Impact Assessment}: Evaluating the social and technical implications of code generation to minimize negative effects. \\
     \textbf{Regulatory Compliance}: Ensuring that generated code adheres to relevant laws (e.g., copyright) and industry regulations.
    %  \textbf{Sustainability}: Promoting environmentally conscious practices in model training and deployment to reduce carbon footprint.
} \\
\midrule
\textbf{Efficiency} & 
\makecell[l]{
     \textbf{Model Optimization}: Streamlining models to reduce computational load and improve speed. \\
     \textbf{Prompt Engineering}: Designing effective prompts to generate accurate code efficiently. \\
     \textbf{Resource Management}: Allocating computational resources wisely to balance speed and cost. \\
     \textbf{Inference Optimization}: Enhancing the inference process to quickly generate code with minimal latency. \\
     \textbf{Parallel Processing}: Utilizing parallelism to speed up code generation tasks. \\
     \textbf{Caching Mechanisms}: Implementing caching to reuse previous results and reduce redundant computations. \\
     \textbf{Evaluation Metrics}: Using precise metrics to assess and improve the efficiency of code outputs.
} \\
\midrule
\textbf{Safety} & 
\makecell[l]{
%      \textbf{Safety}: Directly relates to the system’s ability to ensure protection. \\
%      \textbf{Robustness}: Affects the system's capability to remain secure under various conditions. \\
%      \textbf{Generating Harmful or Useless Code}: Impacts the system’s safety and reliability. \\
%      \textbf{Hateful Code Comments}: Can affect the safety culture and user experience.
     \textbf{Input Validation}: Ensuring inputs (prompts) are safe and sanitized to prevent malicious exploitation. \\
     \textbf{Security Audits}: Regularly reviewing generated code for vulnerabilities and potential exploits. \\
     \textbf{Monitoring and Logging}: Keeping track of generation outputs to quickly identify and address safety issues. \\
     \textbf{User Access Control}: Limiting access to generation capabilities to trusted users to minimize risk. \\
     \textbf{Continuous Updates}: Regularly updating models with the latest safety protocols and security patches. \\
     \textbf{Ethical Guidelines}: Implementing ethical standards to guide safe and responsible code generation.
} \\
\midrule
\textbf{Trustworthiness} & 
\makecell[l]{
%      \textbf{Fairness}: Ensures the system treats all users equitably, which is fundamental to trust.\\
%      \textbf{Bias in Evaluation}: Affects the system's fairness, impacting its trustworthiness.
     \textbf{Reliability}: Ensuring that generated code consistently meets functional requirements and performs as expected. \\
     \textbf{Transparency}: Providing clear explanations of how code is generated to build user confidence. \\
     \textbf{Verification and Testing}: Using rigorous testing frameworks to ensure the generated code accuracy and reliability. \\
    % the correctness and reliability of the generated code. \\
     \textbf{Bias Mitigation}: Actively working to identify and reduce biases in code generation to ensure fairness and impartiality. \\
     \textbf{User Feedback Integration}: Continuously incorporating user feedback to refine and improve code generation processes. \\
     \textbf{Documentation}: Providing comprehensive documentation for generated code to enhance understanding and trust.
    } \\
\bottomrule
\end{tabular}
}
\end{table}